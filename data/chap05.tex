% -*- coding: utf-8 -*-
% !TEX program = xelatex

\chapter{罗列环境的使用}



\section{普通罗列环境}



\begin{itemize}
    \addtolength{\itemsep}{-.36\baselineskip}%缩小条目之间的间距,下面类似
    \item 大漠孤烟直
    \item 长河落日圆
\end{itemize}




\begin{enumerate}
    \item 从散射理论出发。
    \item 从将共振态视为复哈密顿量本征值出发。
    \item 从共振态与呈指数衰减的波函数有关的观点出发(含时理论)。
\end{enumerate}

\section{进阶罗列环境}


\begin{yklist}{Feshbach共振}
    \item[势形共振] 势形共振是指隧穿势垒的共振,通常特指发生在单个通道。势能曲线有两个极点:一个极小点(平衡点),一个极大点(势垒)。形成这个势垒的原因有多种。例如,高分波离心势项$\frac{\hbar \ell(\ell+1)}{2\mu R^2}$ 会使有效势$ V_{\text{eff}}(R,\ell) $ 在较大核间距处出现一个离心势垒。
    \item[Feshbach共振] Feshbach共振又称成为Fano共振,或者Fano-Feshbach共振,是以物理学家H.~Feshbach和U.~Fano的名字命名\cite{RN1793,RN417,RN38},但是E.~Majorana才是第一个观测到这种现象的人\cite{RN2608}。
\end{yklist}










